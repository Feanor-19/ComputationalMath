\documentclass[a4paper,12pt]{article}
\usepackage[left=1.5cm,right=1.5cm,top=2cm,bottom=2cm]{geometry}
\usepackage{cmap}
\usepackage{mathtext}
\usepackage[T2A]{fontenc}
\usepackage[utf8]{inputenc}
\usepackage[english,russian]{babel}

\usepackage{graphicx}
%\graphicspath{{noiseimages/}}
\usepackage{wrapfig}
\usepackage{tabularx}
\usepackage{hyphenat}
\usepackage{hyperref}
\usepackage{gensymb}
\usepackage[rgb]{xcolor}
\hypersetup{
colorlinks=true,urlcolor=blue
}

\usepackage{adjustbox}

%%% Дополнительная работа с математикой
\usepackage{amsmath,amsfonts,amssymb,amsthm,mathtools} % AMS
\usepackage{icomma} % "Умная" запятая: $0,2$ --- число, $0, 2$ --- перечисление

%% Номера формул
%\mathtoolsset{showonlyrefs=true} % Показывать номера только у тех формул, на которые есть \eqref{} в тексте.

%% Шрифты
\usepackage{euscript}	 % Шрифт Евклид
\usepackage{mathrsfs} % Красивый матшрифт

%% Свои команды
\DeclareMathOperator{\sgn}{\mathop{sgn}}

\usepackage{graphics}
\usepackage{wrapfig}
\usepackage{float}
\usepackage{siunitx} % Required for alignment
\usepackage{subfigure}
\usepackage{multirow}
\usepackage{rotating}
\usepackage{afterpage}
\usepackage[T1,T2A]{fontenc}
\usepackage{caption}
\usepackage[arrowdel]{physics}
\usepackage{booktabs}

\newcommand{\rref}[1]{(\ref{#1})}
\newcommand{\Equip}[3]{{\bf #1:} $\Delta = \pm #2$ \si{#3}

}
\newcommand{\equip}[1]{{\bf #1}

}

\begin{document}

\section{Аннотация}

В данной лабораторной работе исследуются методы решения нелинейных уравнений и нелинейных систем уравнений. Для нескольких отобранных уравнений, систем и их решений получены зависимости убывания невязки от номера итерации, на основе которых проведено небольшое сравнение методов.

\section{Методология}

В данной работе для изучения методов решения нелинейных уравнений и нелинейных систем уравнений выбраны следующие задачи:

\begin{enumerate}
\item Нелинейные уравнения:

\begin{equation}
x^2 - \frac{e^x}{5} = 0
\label{eq:f_1}
\end{equation}

\begin{equation}
x 2^x -1 = 0
\label{eq:f_2}
\end{equation}

\item Нелинейные системы уравнений:

\begin{equation}
\begin{cases}
	x-\cos{y}-3=0, \\
	\cos{x-1}+y-0.5=0
\end{cases}
\label{eq:F_1}
\end{equation}

\begin{equation}
\begin{cases}
(x-1.4)^2 - (y-0.6)^2 - 1=0, \\
4.2x^2 + 8.8y^2 - 1.42=0
\end{cases}
\label{eq:F_2}
\end{equation}

\end{enumerate}

Справочная информация об используемых методах приведена в разделе ``Описание методов'' ниже. Комментарии по конкретной реализации (см. соответстующие Python исходники):
\begin{enumerate}
\item Выбранная точность $\epsilon=10^{-7}$;
\item Ограничение на максимальное число итераций: 1000;
\item В методе простой итерации (далее МПИ) приведение к виду $X = G(X)$ производится путём выбора $G(X) = x - F(X)$;
\item В МПИ условие Липшица проверяется в начальной точке с $q < 0.9$;
\item В обоих методах Ньютона якобиан вычисляется численно, используя формулу разностной численной производной;
\item В обоих методах Ньютона для решение СЛАУ используется алгоритм LU-разложения из предудыщей лабораторной работы.
\end{enumerate}

Для всех уравнений строятся графики ($y = f(x)$ в одномерном случае, кривые $F(x, y) = 0$ в двумерном) и по ним в одномерном случае определяются отрезки, содержащие корни, и приближённые значения искомых решений в многомерном. Для уравнений \eqref{eq:f_1}, \eqref{eq:f_2} используются все методы (для многомерных методов в качестве начального приближения берутся середины найденных отрезков), для систем уравнений \eqref{eq:F_1}, \eqref{eq:F_2} - только предназначенные для систем уравнений. Для каждого решения и каждого метода строится график зависимости невязки от номера итерации. В качестве невязки берётся евклидова норма значения функции в данной точке.

\section{Описание методов}

\subsection{Нелинейные уравнения}

\subsubsection{Метод половинного деления (бисекции)}

\textbf{Тип метода:} Итерационный численный метод решения нелинейных уравнений.

\textbf{Условия применимости:}
\begin{enumerate}
    \item Функция \( f(x) \) должна быть непрерывной на отрезке \([a, b]\)
    \item На концах интервала значения функции должны иметь разные знаки: \( f(a) \cdot f(b) < 0 \)
\end{enumerate}

\textbf{Алгоритм:}
\begin{enumerate}
    \item Выбирается начальный интервал \([a, b]\), содержащий корень
    \item Вычисляется середина интервала: \( c = \frac{a + b}{2} \)
    \item Проверяется знак \( f(c) \):
    \begin{itemize}
        \item Если \( f(a) \cdot f(c) < 0 \), корень находится в \([a, c]\)
        \item Иначе корень в \([c, b]\)
    \end{itemize}
    \item Процесс повторяется с новым интервалом
\end{enumerate}

\textbf{Критерий остановки:}
\[
|b - a| < \varepsilon \quad \text{или} \quad |f(c)| < \varepsilon
\]
где \(\varepsilon\) — заданная точность.

\subsubsection{Метод хорд}

\textbf{Тип метода:} Итерационный метод решения нелинейных уравнений.

\textbf{Условия применимости:}
\begin{enumerate}
    \item Функция \( f(x) \) непрерывна на отрезке \([a, b]\)
    \item \( f(a) \cdot f(b) < 0 \) (на концах отрезка функция принимает значения разных знаков)
    \item Производные \( f'(x) \) и \( f''(x) \) сохраняют знак на отрезке \([a, b]\)
\end{enumerate}

\textbf{Алгоритм:}
\begin{enumerate}
    \item Выбрать начальный отрезок \([a, b]\), где функция меняет знак
    \item Зафиксировать конец отрезка, где знак функции совпадает со знаком второй производной:
    \begin{itemize}
        \item Если \( f(a) \cdot f''(a) > 0 \), то \( a \) — неподвижный конец
        \item Иначе \( b \) — неподвижный конец
    \end{itemize}
    \item Вычислить новое приближение:
    \[
    x = a - \frac{f(a)(b-a)}{f(b)-f(a)}
    \]
    \item Обновить интервал:
    \begin{itemize}
        \item Если \( f(a) \cdot f(x) < 0 \), то \( b = x \)
        \item Иначе \( a = x \)
    \end{itemize}
    \item Повторять шаги 3-4 до достижения заданной точности
\end{enumerate}

\textbf{Критерий остановки:}
\[
|f(x)| < \varepsilon \quad \text{или} \quad |x_{k+1} - x_k| < \varepsilon
\]

\subsection{Нелинейные системы уравнений}

\subsubsection{Метод простой итерации}

\textbf{Тип метода:} Итерационный метод решения систем нелинейных уравнений.

\textbf{Условия применимости:}
\begin{enumerate}
    \item Система должна быть приведена к виду \( X = G(X) \), где \( G \) — сжимающее отображение
    \item Существует константа \( q < 1 \): \( \|G(X_1) - G(X_2)\| \leq q\|X_1 - X_2\| \)
    \item Начальное приближение \( X_0 \) должно находиться в области сходимости
\end{enumerate}

\textbf{Алгоритм:}
\begin{enumerate}
    \item Привести систему \( F(X) = 0 \) к эквивалентному виду \( X = G(X) \)
    \item Выбрать начальное приближение \( X_0 \) и точность \( \varepsilon \)
    \item Выполнять итерации:
    \begin{align*}
        X_{k+1} &= G(X_k) \\
        \|X_{k+1} - X_k\| &< \varepsilon
    \end{align*}
\end{enumerate}

\textbf{Критерий остановки:}
\[
\|X_{k+1} - X_k\| < \varepsilon
\]

\subsubsection{Метод Ньютона (Ньютона-Рафсона)}

\textbf{Тип метода:} Итерационный метод решения систем нелинейных уравнений.

\textbf{Условия применимости:}
\begin{enumerate}
    \item Функция \( F(X) \) должна быть дифференцируема в окрестности решения
    \item Матрица Якоби \( J(X) \) не должна быть вырожденной в решении
    \item Начальное приближение \( X_0 \) должно быть достаточно близко к корню
\end{enumerate}

\textbf{Алгоритм:}
\begin{enumerate}
    \item Задать начальное приближение \( X_0 \), точность \( \varepsilon \), максимальное число итераций \( N \)
    \item Для \( k = 0, 1, \dots, N \):
    \begin{enumerate}
        \item Вычислить \( F_k = F(X_k) \) и матрицу Якоби \( J_k = J(X_k) \)
        \item Решить систему линейных уравнений: \( J_k \cdot \Delta X_k = -F_k \)
        \item Обновить решение: \( X_{k+1} = X_k + \Delta X_k \)
        \item Проверить условие остановки
    \end{enumerate}
\end{enumerate}

\textbf{Критерий остановки:}
\[
\|\Delta X_k\| < \varepsilon \quad \text{или} \quad \|F(X_{k+1})\| < \varepsilon
\]

\subsubsection{Модифицированный метод Ньютона}

\textbf{Тип метода:} Итерационный метод решения систем нелинейных уравнений.

\textbf{Условия применимости:}
\begin{enumerate}
    \item Функция \( F(X) \) должна быть дифференцируемой в окрестности решения
    \item Матрица Якоби \( J(X) \) должна быть обратима в начальном приближении
    \item Начальное приближение \( X_0 \) должно быть достаточно близко к решению
\end{enumerate}

\textbf{Алгоритм:}
\begin{enumerate}
    \item Выбрать начальное приближение \( X_0 \) и точность \( \varepsilon \)
    \item Вычислить матрицу Якоби \( J_0 = J(X_0) \) один раз в начальной точке
    \item Для \( k = 0, 1, 2, \dots \) до достижения точности:
    \begin{enumerate}
        \item Вычислить \( F_k = F(X_k) \)
        \item Решить систему линейных уравнений: \( J_0 \Delta X_k = -F_k \)
        \item Обновить решение: \( X_{k+1} = X_k + \Delta X_k \)
        \item Проверить условие остановки
    \end{enumerate}
\end{enumerate}

\textbf{Критерий остановки:}
\[
\|F(X_{k+1})\| < \varepsilon \quad \text{или} \quad \|\Delta X_k\| < \varepsilon
\]

\textbf{Особенность:} Матрица Якоби вычисляется только один раз, что уменьшает вычислительную сложность, но может снизить скорость сходимости.

\section{Исследование}

\subsection{Нелинейные уравнения}

График для уравнения \eqref{eq:f_1} приведён на рис. \ref{graf:plot1d_1}. По нему определяются отрезки, содержащие все корни:

\begin{enumerate}
\item $[-1, 0]$
\item $[0, 1]$
\item $[4, 5]$
\end{enumerate}

 \begin{figure}[H]
 	\centering
 	\includegraphics[width=0.6\linewidth]{images/plot1d_1.png}
 	\caption{}
 	\label{graf:plot1d_1}
 \end{figure}
 
 На рис. \ref{graf:resds_1d_1_all} приведены все три найденные решения и графики убывания невязки в зависимости от номера итерации. МПИ сошёлся только в одной точке, в остальных нарушалось условие Липшица.
 
  \begin{figure}[H]
 	\centering
 	\includegraphics[width=1.15\linewidth]{images/resds_1d_1_all.png}
 	\caption{}
 	\label{graf:resds_1d_1_all}
 \end{figure}
 
 График для уравнения \eqref{eq:f_2} приведён на рис. \ref{graf:plot1d_2}. По нему определяются отрезки, содержащие все корни:

\begin{enumerate}
\item $[0, 1]$
\end{enumerate}

 \begin{figure}[H]
 	\centering
 	\includegraphics[width=0.6\linewidth]{images/plot1d_2.png}
 	\caption{}
 	\label{graf:plot1d_2}
 \end{figure}
 
  На рис. \ref{graf:resds_1d_2_all} приведено найденное решение и графики убывания невязки в зависимости от номера итерации. МПИ в данной точке не сошёлся.
 
  \begin{figure}[H]
 	\centering
 	\includegraphics[width=1.15\linewidth]{images/resds_1d_2_all.png}
 	\caption{}
 	\label{graf:resds_1d_2_all}
 \end{figure}

\subsection{Нелинейные системы уравнений}

График для системы уравнений \eqref{eq:F_1} приведён на рис. \ref{graf:plot2d_1}. По нему определяется приближённое решение:

\begin{enumerate}
\item $(3.5, 1.5)$
\end{enumerate}

 \begin{figure}[H]
 	\centering
 	\includegraphics[width=0.6\linewidth]{images/plot2d_1.png}
 	\caption{}
 	\label{graf:plot2d_1}
 \end{figure}
 
 На рис. \ref{graf:resds_2d_1} приведено найденное решение и графики убывания невязки в зависимости от номера итерации.
 
  \begin{figure}[H]
 	\centering
 	\includegraphics[width=1.15\linewidth]{images/resds_2d_1.png}
 	\caption{}
 	\label{graf:resds_2d_1}
 \end{figure}
 
 График для системы уравнений \eqref{eq:F_2} приведён на рис. \ref{graf:plot2d_2}. По нему определяются приближённые решения:

\begin{enumerate}
\item $(0, -0.4)$
\item $(0.4, 0.4)$
\end{enumerate}

 \begin{figure}[H]
 	\centering
 	\includegraphics[width=0.6\linewidth]{images/plot2d_2.png}
 	\caption{}
 	\label{graf:plot2d_2}
 \end{figure}
 
 На рис. \ref{graf:resds_2d_2} приведено найденное решение и графики убывания невязки в зависимости от номера итерации. МПИ не сошёлся в обеих точках.
 
  \begin{figure}[H]
 	\centering
 	\includegraphics[width=0.8\linewidth]{images/resds_2d_2.png}
 	\caption{}
 	\label{graf:resds_2d_2}
 \end{figure}

\section{Обсуждение результатов}

По полученным результатам можно сделать несколько выводов:

\begin{enumerate}
\item Из всех рассмотренных методов МПИ - единственный, который сходился не во всех точках. В одномерном случае он сошёлся быстрее всех методов, кроме метода Ньютона. Однако в многомерном ему потребовалось на порядок больше итераций, чем двум другим многомерным методам. 
\item Модифицированный метод Ньютона ожидаемо везде сходится медленнее, чем обычный метод Ньютона, но не более чем в пять раз. Также стоит отметить, что он требует вычисление якобиана только единожды. Из-за этого скорость сходимости может сильно зависеть от выбора начального приближения, а также может быть лучше общая производительность, чем у метода Ньютона. Обе гипотезы требуют отдельной проверки. 
\item На рассмотренных нелинейных уравнениях метод хорд показал лучшие результаты, чем метод бисекций. Метод хорд имеет более строгие условия применимости, но их нетрудно удовлетворить, если выбирать начальный отрезок на основе графика.
\item Метод Ньютона сходится очень быстро (всего за несколько итераций) и в многомерном, и в одномерном (как более частном) случае. Такая скорость сходимости приводит к тому, что точность решение лучше заданного $\epsilon$ на несколько порядков, чего не наблюдается у других методов. Но нельзя забывать про значительную относительную вычислительную сложность.
\end{enumerate}

\section{Вывод}

\begin{enumerate}
\item Метод простой итерации не рекомендуется использовать из-за слишком строгих условий сходимости, которым трудно удовлетворить.
\item Метод Ньютона отлично себя показал во всех случаях, включая одномерный, но если есть сильные ограничения по времени и достаточно большая размерность задачи, нужно либо тщательно оптимизировать вычисление якобиана и алгоритм решения СЛАУ, либо использовать модифицированный метод Ньютона.
\item Для проверки озвученных выше гипотез, касающихся модифицированного метода Ньютона (про выигрыш в общей производительности и зависимость скорости сходимости от начального приближения) требуются дополнительные исследования.
\end{enumerate}

\end{document}
