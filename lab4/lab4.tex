\documentclass[a4paper,12pt]{article}
\usepackage[left=1.5cm,right=1.5cm,top=2cm,bottom=2cm]{geometry}
\usepackage{cmap}
\usepackage{mathtext}
\usepackage[T2A]{fontenc}
\usepackage[utf8]{inputenc}
\usepackage[english,russian]{babel}

\usepackage{graphicx}
%\graphicspath{{noiseimages/}}
\usepackage{wrapfig}
\usepackage{tabularx}
\usepackage{hyphenat}
\usepackage{hyperref}
\usepackage{gensymb}
\usepackage[rgb]{xcolor}
\hypersetup{
colorlinks=true,urlcolor=blue
}

\usepackage{adjustbox}

%%% Дополнительная работа с математикой
\usepackage{amsmath,amsfonts,amssymb,amsthm,mathtools} % AMS
\usepackage{icomma} % "Умная" запятая: $0,2$ --- число, $0, 2$ --- перечисление

%% Номера формул
%\mathtoolsset{showonlyrefs=true} % Показывать номера только у тех формул, на которые есть \eqref{} в тексте.

%% Шрифты
\usepackage{euscript}	 % Шрифт Евклид
\usepackage{mathrsfs} % Красивый матшрифт

%% Свои команды
\DeclareMathOperator{\sgn}{\mathop{sgn}}

\usepackage{graphics}
\usepackage{wrapfig}
\usepackage{float}
\usepackage{siunitx} % Required for alignment
\usepackage{subfigure}
\usepackage{multirow}
\usepackage{rotating}
\usepackage{afterpage}
\usepackage[T1,T2A]{fontenc}
\usepackage{caption}
\usepackage[arrowdel]{physics}
\usepackage{booktabs}

\newcommand{\rref}[1]{(\ref{#1})}
\newcommand{\Equip}[3]{{\bf #1:} $\Delta = \pm #2$ \si{#3}

}
\newcommand{\equip}[1]{{\bf #1}

}

\begin{document}

\section{Методология}

В данной работе сравниваются три метода на примере задачи экстраполяции таблично заданной функции ``недалеко'' от границы определения. Под ``недалеко'' имеется в виду в пределах 10\% от диапазона узлов за границей области определения. В качестве функции взята \href{https://ru.wikipedia.org/wiki/Перепись_населения_США}{численность населения США в 1900-2020 гг}. Экстраполяция проводится по всем годам, кроме последнего, а затем сравнивается вычисленное значение в 2020г с настоящим. 

Из трёх используемых методов два являются интерполяционными методами. Несмотря на это, они всё равно могут быть использованы для экстраполяции при выполнении определённых условий (см. ``Описание методов''). 

\textbf{Примечание:}  В методе наименьших квадратов используется полином максимальной возможной степени для данного количества узлов: $n=11$.

\section{Описание методов}

В данном разделе описаны три метода приближения значений функций: классическая полиномиальная интерполяция по Ньютону, сплайн-интерполяция и метод наименьших квадратов. Каждый метод представлен с указанием типа, условий применимости, алгоритма и особенностей.

\subsection{Классическая полиномиальная интерполяция по Ньютону}

\begin{itemize}
    \item \textbf{Тип метода:} Интерполяция (точное прохождение через заданные узлы) с поддержкой экстраполяции.
    \item \textbf{Характеристики:}
    \begin{itemize}
        \item Полиномиальная (использует алгебраический полином).
        \item Точечная (использует таблицу значений $(x_i, f_i)$).
        \item Универсальная (узлы не обязательно равноотстоящие).
        \item Форма Ньютона удобна для добавления новых узлов без пересчёта всех коэффициентов.
        \item Поддерживает экстраполяцию: позволяет вычислять значения за пределами интервала узлов.
    \end{itemize}
\end{itemize}

\subsubsection*{Условия применимости и ограничения}
\begin{enumerate}
    \item Задана таблица из $(n+1)$ точки $(x_i, f_i)$, где $i = 0, 1, \dots, n$.
    \item Все узлы $x_i$ различны (условие единственности интерполяционного полинома).
    \item Функция $f(x)$ предполагается непрерывной и достаточное число раз дифференцируемой на интервале, содержащем узлы (для оценки погрешности).
    \item \textbf{Экстраполяция:} Метод позволяет вычислять значения за пределами интервала узлов, однако при этом:
    \begin{itemize}
        \item Погрешность экстраполяции обычно значительно превышает погрешность интерполяции.
        \item При удалении от области узлов погрешность растёт быстро (иногда экспоненциально).
        \item Особенно опасна экстраполяция полиномами высокой степени (более 5-6).
        \item Рекомендуется экстраполяция ``недалеко'' за границы области узлов (в пределах 10-20\% от диапазона узлов).
    \end{itemize}
\end{enumerate}

\subsubsection*{Алгоритм метода}
Пусть заданы узлы $x_0, x_1, \dots, x_n$ и значения $f_i = f(x_i)$.
\begin{enumerate}
    \item \textbf{Построение разделённых разностей:}
    \begin{itemize}
        \item Разделённые разности нулевого порядка: $f[x_i] = f_i$.
        \item Разделённые разности первого порядка:
        \[
        f[x_i, x_{i+1}] = \frac{f[x_{i+1}] - f[x_i]}{x_{i+1} - x_i}.
        \]
        \item Разделённые разности $k$-го порядка (рекуррентно):
        \[
        f[x_i, x_{i+1}, \dots, x_{i+k}] = \frac{f[x_{i+1}, \dots, x_{i+k}] - f[x_i, \dots, x_{i+k-1}]}{x_{i+k} - x_i}.
        \]
    \end{itemize}
    \item \textbf{Построение интерполяционного полинома Ньютона:}
    \[
    P_n(x) = f[x_0] + f[x_0, x_1](x - x_0) + f[x_0, x_1, x_2](x - x_0)(x - x_1) + \dots + f[x_0, x_1, \dots, x_n](x - x_0)(x - x_1)\dots(x - x_{n-1}).
    \]
    Или в компактной форме:
    \[
    P_n(x) = \sum_{k=0}^{n} \left( f[x_0, \dots, x_k] \cdot \prod_{j=0}^{k-1} (x - x_j) \right).
    \]
    \item \textbf{Вычисление значения в произвольной точке $x$:}
    Подставить $x$ в построенный полином $P_n(x)$. При этом:
    \begin{itemize}
        \item Если $x \in [\min(x_i), \max(x_i)]$ --- это интерполяция.
        \item Если $x < \min(x_i)$ или $x > \max(x_i)$ --- это экстраполяция.
    \end{itemize}
\end{enumerate}

\subsubsection*{Погрешность метода}
Для любой точки $x$ (включая область экстраполяции) справедливо:
\[
f(x) - P_n(x) = \frac{f^{(n+1)}(\xi)}{(n+1)!} \cdot \omega_n(x), \quad \omega_n(x) = \prod_{i=0}^{n} (x - x_i),
\]
где $\xi$ --- некоторая точка из наименьшего интервала, содержащего все узлы $x_i$ и точку $x$.

\subsubsection*{Особенности экстраполяции}
\begin{itemize}
    \item При экстраполяции множитель $\omega_n(x)$ растёт быстрее, чем при интерполяции.
    \item Для экстраполяции ``вперёд'' (за $x_n$) рекомендуется упорядочить узлы по возрастанию и использовать полином в форме Ньютона для интерполяции вперёд.
    \item Для экстраполяции ``назад'' (до $x_0$) можно использовать ту же форму, но более устойчивые результаты даёт переупорядочивание узлов в обратном порядке.
\end{itemize}

\subsection{Сплайн-интерполяция с экстраполяцией}

\begin{itemize}
    \item \textbf{Тип метода:} Интерполяционный (значения сплайна совпадают с заданными точками) с поддержкой экстраполяции.
    \item \textbf{Характеристики:}
    \begin{itemize}
        \item Гладкая аппроксимация (непрерывны функция и её первые две производные).
        \item Кубические сплайны (на каждом отрезке используется полином 3-й степени).
    \end{itemize}
\end{itemize}

\subsubsection*{Условия применимости}
\begin{enumerate}
    \item Узлы \(x_i\) должны быть \textbf{строго возрастающими}: \(x_0 < x_1 < \dots < x_n\).
    \item Количество точек \(n+1 \geq 2\).
    \item Тип граничных условий должен быть определён. В реализации используется \textbf{естественный сплайн} (\(S''(x_0)=S''(x_n)=0\)).
    \item Экстраполяция рекомендуется только вблизи границ интервала (\(x \in [x_0 - \delta, x_n + \delta]\) с малым \(\delta\)).
\end{enumerate}

\subsubsection*{Алгоритм метода}
Пусть заданы узлы \(x_0, x_1, \dots, x_n\) и значения \(f_i = f(x_i)\). На каждом отрезке \([x_i, x_{i+1}]\) сплайн имеет вид:
\[
S_i(x) = a_i + b_i(x - x_i) + c_i(x - x_i)^2 + d_i(x - x_i)^3.
\]

\noindent \textbf{Шаг 1: Вычисление разностей}
\[
h_i = x_{i+1} - x_i, \quad \Delta_i = \frac{f_{i+1} - f_i}{h_i}, \quad i = 0, \dots, n-1.
\]

\noindent \textbf{Шаг 2: Решение системы для моментов \(M_i = S''(x_i)\)}\\
Система для естественного сплайна (\(M_0 = M_n = 0\)):
\[
\begin{cases}
\mu_i M_{i-1} + 2M_i + \lambda_i M_{i+1} = d_i, \quad i=1,\dots,n-1,\\
M_0 = M_n = 0,
\end{cases}
\]
где
\[
\mu_i = \frac{h_{i-1}}{h_{i-1} + h_i}, \quad \lambda_i = 1 - \mu_i, \quad d_i = 6 \frac{\Delta_i - \Delta_{i-1}}{h_{i-1} + h_i}.
\]

\noindent \textbf{Шаг 3: Трёхдиагональная прогонка (метод Томаса)}
\begin{enumerate}
    \item \textbf{Прямой ход} (вычисление прогоночных коэффициентов):
    \[
    \alpha_1 = -\frac{\lambda_1}{2}, \quad \beta_1 = \frac{d_1}{2},
    \]
    \[
    \alpha_i = -\frac{\lambda_i}{2 + \mu_i \alpha_{i-1}}, \quad \beta_i = \frac{d_i - \mu_i \beta_{i-1}}{2 + \mu_i \alpha_{i-1}}, \quad i=2,\dots,n-1.
    \]
    \item \textbf{Обратный ход}:
    \[
    M_{n-1} = \beta_{n-1}, \quad M_i = \alpha_i M_{i+1} + \beta_i, \quad i=n-2,\dots,1.
    \]
\end{enumerate}

\noindent \textbf{Шаг 4: Вычисление коэффициентов сплайна}\\
Для каждого отрезка \(i=0,\dots,n-1\):
\[
a_i = f_i, \quad c_i = \frac{M_i}{2},
\]
\[
d_i = \frac{M_{i+1} - M_i}{6h_i}, \quad b_i = \Delta_i - \frac{h_i(2M_i + M_{i+1})}{6}.
\]

\noindent \textbf{Шаг 5: Вычисление значения с экстраполяцией}\\
Для заданного \(x\):
\begin{enumerate}
    \item Если \(x < x_0\), используем первый отрезок: \(i = 0\).
    \item Если \(x > x_n\), используем последний отрезок: \(i = n-1\).
    \item Иначе находим индекс \(i\) такой, что \(x_i \leq x \leq x_{i+1}\).
\end{enumerate}
Затем вычисляем:
\[
S(x) = a_i + b_i(x - x_i) + c_i(x - x_i)^2 + d_i(x - x_i)^3.
\]

\subsubsection*{Особенности экстраполяции}
\begin{itemize}
    \item \textbf{Естественный сплайн} имеет нулевую вторую производную на границах, что обеспечивает линейное поведение при экстраполяции на бесконечности.
    \item На практике кубический сплайн может давать разумные значения при экстраполяции на расстояниях, не превышающих шаг интерполяции.
    \item Качество экстраполяции ухудшается по мере удаления от границ интервала.
    \item Рекомендуется использовать экстраполяцию только вблизи границ \([x_0 - h_0, x_n + h_{n-1}]\).
\end{itemize}

\subsection{Метод наименьших квадратов (МНК)}

\begin{itemize}
    \item \textbf{Тип метода:} Аппроксимация (сглаживание). МНК не является интерполяцией, так как не требует точного прохождения функции через узловые точки. Он предназначен для построения приближающей функции, \textit{наилучшим образом} описывающей общую тенденцию данных, что особенно полезно при наличии случайных погрешностей в измерениях.
    \item \textbf{Суть метода:} Метод позволяет подобрать функцию заданного вида (например, полином степени $m$), которая минимизирует сумму квадратов отклонений значений этой функции от заданных значений в узлах.
\end{itemize}

\subsubsection*{Условия применимости}
\begin{itemize}
    \item Количество точек $n$ должно быть больше количества искомых параметров $m+1$ (система переопределена).
    \item Искомая зависимость (вид функции $F$) должна быть выбрана априори на основе анализа природы данных или задачи (линейная, полиномиальная, экспоненциальная и т.д.).
    \item Для линейного по параметрам случая ($F(x, \vec{c}) = \sum_{k=0}^{m} c_k \varphi_k(x)$), где $\varphi_k(x)$ --- базисные функции (например, $x^k$), метод сводится к решению \textit{системы нормальных уравнений} или линейной задачи. Для нелинейных моделей требуется применение итерационных методов оптимизации.
\end{itemize}

\subsubsection*{Алгоритм (для полиномиальной модели $P_m(x) = c_0 + c_1 x + \dots + c_m x^m$)}
\begin{enumerate}
    \item \textbf{Формулировка задачи.} Даны $n$ пар чисел $(x_i, y_i)$. Требуется найти коэффициенты полинома $m$-й степени ($m < n-1$), минимизирующие сумму $S = \sum_{i=1}^n (y_i - P_m(x_i))^2$.
    \item \textbf{Составление системы нормальных уравнений.} Необходимое условие минимума $S$ --- равенство нулю частных производных $\frac{\partial S}{\partial c_k} = 0$ для $k=0,\dots,m$. Это приводит к системе $(m+1)$ линейных уравнений:
    \[
    \sum_{j=0}^{m} c_j \sum_{i=1}^{n} x_i^{j+k} = \sum_{i=1}^{n} y_i x_i^{k}, \quad k = 0, 1, \dots, m.
    \]
    В матричной форме: $\mathbf{A}^T\mathbf{A} \vec{c} = \mathbf{A}^T\vec{y}$, где
    \[
    A_{ij} = x_i^{j-1} \quad (i=1..n, j=1..m+1), \quad \vec{y} = (y_1, \dots, y_n)^T.
    \]
    \item \textbf{Решение системы.} Решается система относительно вектора коэффициентов $\vec{c} = (c_0, c_1, \dots, c_m)^T$. Для устойчивости при больших $m$ часто используют не прямое решение нормальных уравнений, а методы на основе QR- или SVD-разложения матрицы $\mathbf{A}$.
    \item \textbf{Получение аппроксимирующей функции.} Искомая функция --- полином $P_m(x)$ с найденными коэффициентами.
\end{enumerate}
* QR-разложение (ортогонально-треугольное) и SVD (сингулярное разложение) --- численно устойчивые альтернативы для решения задачи МНК.


\section{Исследование}

Полученный график представлен на рис. \ref{graf}. Кривая метода Ньютона ушла в глубоко отрицательные значения и поэтому её значение в узле 2020г. не попало на график. 

  \begin{figure}[H]
 	\centering
 	\includegraphics[width=0.8\linewidth]{images/graf.png}
 	\caption{}
 	\label{graf}
 \end{figure}
 
 Полученные относительные погрешности в целевой точке (2020г.):
 
 $$
 r_{newton} = 545 \% 
 $$
 
 $$
 r_{spline} = 1.4 \%
 $$
 
 $$
 r_{mnk} = 2.0 \%
 $$

\section{Обсуждение результатов}

Из полученных результатов очевидно, что метод Ньютона не применим для экстраполяции даже в 10 \% от диапазона узлов. Слишком большая степень полинома при 12 узлах привела к выбросу даже между последним и предпоследним узлом. 

Метод кубических сплайнов дал наименьшую среди всех методов относительную ошибку экстраполяции несмотря на то, что является интерполяционным методом. 

\end{document}
